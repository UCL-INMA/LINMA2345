\ifx \globalmark \undefined %% This is default.
	\documentclass[a4paper,12pt]{report}

%% The two first options are wrong in english: they do not agree with some standard english convenions
%\usepackage[utf8] {inputenc}	% LaTeX, comprends les accents !
\usepackage[french, english] {babel}	% langue principale
\usepackage[ansinew]{inputenc}

\usepackage[T1]{fontenc}		% Police contenant les caractères françai
\usepackage{lmodern}			% plus beau

\usepackage[a4paper]{geometry}
\geometry{hscale=0.75,vscale=0.8,centering}

\usepackage[hidelinks]{hyperref}
\usepackage{amsfonts} % Equations etc
\usepackage{amsmath}
\usepackage{mathtools}
\usepackage{amssymb}
\usepackage{enumerate}
\usepackage{enumitem}
\usepackage[pdftex]{graphicx} %Images dans le PDF
\usepackage{epstopdf}
\usepackage{color}
\usepackage{chessboard}
\usepackage{booktabs,tabularx,multirow}
\usepackage[babel=true,kerning=true]{microtype}
\usepackage{tikz,pgfplots}
\usepackage{wrapfig}

\usetikzlibrary{shapes,arrows,positioning}
\usetikzlibrary{calc}
\usetikzlibrary{decorations}
\usetikzlibrary{decorations.text}

\tikzstyle{noeud-std}=[draw,fill=black,circle,inner sep=0pt,minimum size=7pt]% 7pt est la taille des cercles noirs



\usepackage{color}


\newcommand{\reels}{\mathbb{R}}


% Layout style change by Romain, 2015/05/19
%\setlength{\parindent}{0pt}
\setlength{\parskip}{1ex plus 0.5ex minus 0.2ex}

	\begin{document} %% Crashes if put after (one of the many mysteries of LaTeX?).
\else

\fi



\chapter*{Intro}
Since the birth of humanity, Mathematics have been the central tool for scientists and engineers.  Arguably, Mathematics have not only been a language to express concepts and techniques developed in the different scientific fields, but it has been very often a key motor of the development of these fields, thriving on, and at the same time fertilizing the scientific discipline in question.  This is strikingly the case in decision and game theory.

\paragraph{The birth of a new discipline}

Unlike Mechanics, Chemistry, or other traditional sciences, it was not before the middle of the 20th century that a firm Science of Decision making was laid down.  It is not clear why Decision Theory appeared so late\footnote{Needless to say, this statement is presumptuous, and questions on how to make decisions rationally had already been raised before, notably by Daniel Bernoulli in his 1738 treatise.}, but there are some clear explanations why it appeared by then: computers made their apparition at that moment, unlatching a power that allowed to tackle (some of) the extremely heavy computations needed in Decision Theory.  Interestingly, computers themselves are automatic entities who have to take decisions on the fly, at extremely high pace (which instruction should it realize first?, where should it store its product?,...).  With this in mind, it is not surprising that Von Neuman, one of the two founding fathers of Decision Theory (with O. Morgenstern, also at Princeton's famous Institute for Advanced Study) was also one of the main designers of the first computers.
The middle of the 20th century was also a period where humanity started facing (decision) problems whose scale was reaching dimensions beyond the abilities of the human brain: companies were reaching unprecedented sizes; transport, storage, infrastructure supply were becoming critical problems in our massively populated cities/countries, and technologies were enabling various different solutions for these situations, thus implying decision problems to tackle them at best.  Not to mention WW2, which was really a trigger in the development of mathematical techniques for decision making.
Let us be fair and acknowledge the genius of some mathematicians of that time as another reason for the outcome of this paradigm shift. ``\emph{Beautiful minds}'' like John Von Neumann, or John Nash, where instrumental in these developments.\\
Suddenly, people were understanding that decision could be modeled and optimized mathematically, and that it satisfied fundamental laws, just like two co-prime numbers have to follow Euler's theorem, or the Moon gravitating around the Earth has to follow Kepler's laws.

\paragraph{A journey from rationality to bargaining}

In this course, our goal will be to model, analyze, predict, and prescribe the behavior of agents that have to take decisions in order to optimize their outcome, while these decisions affect the others' welfare too.  \\
We will first model a very general situation, and focus on the decision making problem for a single rational and intelligent agent. We will observe that, after all, we can restrict ourselves to an idealized situation where agents are just trying to optimize their expected ``\emph{payoff}'': this is the fundamental Theorem of Decision Theory. \\
We will then study what happens when two such agents have to take concomitant decisions, with the critical fact that one's decision can have impact on the other's payoff. These situations are referred to as ``\emph{games}'', and will lead us to the celebrated concept of \emph{Nash Equilibrium}.
We will then analyze the impact of temporality on decision making, and will propose refinements of our notion of Equilibria by pushing the concept of rationality as far as we can.  \\
As a third step, we will think as engineers, and, rather than still restricting the range of possible outcomes of a game, we will ask ourselves the question of by which mechanism we can modify these sets of equilibria.  This will take us to the notion of 1) \emph{repeated games}, where repetition is the mechanism that enables this; 2) \emph{to the notion of correlated equilibria}, where a third party, or ``\emph{mediator}'', is the key mechanism; finally we will 3) analyze a particular case of games, namely auctions, where we will investigate several possible mechanisms in order to favor some desired behavior for the players.  This will give us a glimpse at ``\emph{mechanism design}'' problems, which are still an important field of research nowadays.  \\
In the fourth part of the course, we will finish our journey by looking at ``\emph{bargaining problems}'': we will take a step back, and look at all the possible outcomes of a given game (these possible outcomes can for instance be the result of any solution concept previously analyzed in the course); we will ask ourselves the question: \emph{``what if, among all the equilibria, the agents have to agree on the most reasonable outcome?''}.  In other words, among all the possible outcomes, how can we single out a unique one, which will be the 'most fair' (for whatever it means)?

Game Theory is an \emph{axiomatic theory}.  By this, we mean that all the results and concepts that we will see in the course rely on \emph{axioms}, which are a priori postulates on the agents' behaviour.  The concepts that we want to study (several of them we already mentioned here in this introduction) are extremely vague: fairness, rationality, intelligence, bargaining, etc...  If I decide to go to a casino tonight (where my expected gains are strictly negative), rather than staying at home (where my expected gains are zero), am I intelligent?  If I choose to take ``\emph{LPSP1001: Introduction to Psychology}'' rather than ``\emph{LINMA 2345: Game Theory}'', am I rational?  Context is key in answering these questions, and it looks very ambitious to hope to model in a mathematical way the behavior of human agents, let alone predict it.  \\
The proper way to get out of this conundrum is to adopt an axiomatic approach: we will be very careful, at the beginning of this course, to introduce a number (as small as possible) of axioms, that is, mathematical properties that we will not question, and which describe the behavior of rational and intelligent agents.  Needless to say, there are situations where these axioms will be violated; but watch this: if we want to assess whether our mathematical results are of application in some given situation, the only thing we will have to check is this tiny set of elementary properties; and, as we will see, this is remarkable how such a little set of assumptions can have huge consequences.  In fact, every time we will try to model a new vague concept (like, later in the course, the notion of ``\emph{fair bargaining}''), our reflex, as mathematicians, will be to introduce new axioms in order to properly define what we mean by it, thereby laying the ground for a mathematical treatment of this vague notion. \\This axiomatic approach is both the strength and the weakness of Game Theory: on the one hand, one can argue that these axioms are never completely satisfied, and this would invalidate all the results that we will work so hard to obtain; but on the other hand, we will see how powerful they are, and we will perfectly master at every point of the course the range of application of our results: they are completely valid, as soon as the axioms are satisfied.  After all, axioms are nothing but a mathematical model of the true world, and as every model, it has its limits; just like a linear approximation of the effect of an inductance on an electrical circuit, or the assumption that a flow is laminar in Fluid Mechanics.



Today, Decision and Game Theory are more central than ever, with large-scale, multi-agent infrastructures where agents make decisions in an embedded and decentralized way, by then having a critical impact on the global behavior of the system.  Think of the smart grid, where agents decide to produce, consume, sell or buy electricity.  Think of large social networks, where people buy, or consume ads; bid for optimizing their advertising investment; ``like'' ads, and surf in the network, thereby generating a virtual payoff for the advertisers.  Think also of high frequency trading, where several competing computers take decisions, which affect the behavior of the markets in a close loop fashion.\\   Often, these situations are adversary and one is interested in \emph{``the cost of anarchy''}, or \emph{``resilience guarantees''}.  All these problems and notions can only be analyzed with a careful model of how agents behave, and a deep understanding of the potential phenomena occurring when multiple agents make decisions which not only impact their welfare, but also their fellow agents' welfare.  \\
\emph{This is exactly what this course is about.}


\paragraph{Thanks}
I want to express my deep gratitude to Romain Hollanders, Sebastian Stich, and Matthew Philippe for working hard on the first versions of these notes. Many pedagogical ideas presented in these notes come from them.  In particular, I want to stress the amazing work achieved by Matthew, who was leading the project from the first drafts to the final version, not only in writing, but also for many higher level ideas.  Thanks a lot!

\label{chap:intro}




\ifx \globalmark \undefined %% This is default.
\bibliographystyle{plain}
\bibliography{../gametheorybibliography}
	\end{document}
\else

\fi




