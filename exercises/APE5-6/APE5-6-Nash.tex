\documentclass{../ape}

\usepackage{../../linma2345}

\begin{document}

\session{5 \& 6}{Nash equilibria}

\section{}
Find all Nash equilibria in the following games expressed in normal form. For each equilibrium, compute the expected payoff of the players. 

\begin{enumerate}
  \item
    \label{item:first}
	\begin{center}
		\begin{tabular}[h!]{l|ccccc}
			&& \Large{$x_2$} && \Large{$y_2$} & \\
			\hline
			\Large{$x_1$} && \Large{2,1} && \Large{1,2} & \\
			\Large{$y_1$} && \Large{1,5} && \Large{2,1} & 
		\end{tabular} 
	\end{center}
  \item
	\vspace{-.5cm}
	\begin{center}
		\begin{tabular}[h!]{l|ccccc}
			&& \Large{$x_2$} && \Large{$y_2$} & \\
			\hline
			\Large{$x_1$} && \Large{3,7} && \Large{6,6} & \\
			\Large{$y_1$} && \Large{2,2} && \Large{7,3} & 
		\end{tabular} 
	\end{center}
  \item
	\begin{center}
		\begin{tabular}[h!]{l|ccccc}
			&& \Large{$x_2$} && \Large{$y_2$} & \\
			\hline
			\Large{$x_1$} && \Large{7,3} && \Large{6,6} & \\
			\Large{$y_1$} && \Large{2,2} && \Large{3,7} & 
		\end{tabular} 
	\end{center}
  \item
	\begin{center}
		\begin{tabular}[h!]{l|ccccccc}
			&& \Large{$x_2$} && \Large{$y_2$} && \Large{$z_2$} & \\
			\hline
			\Large{$x_1$} && \Large{0,4} && \Large{5,6} && \Large{8,7} & \\
			\Large{$y_1$} && \Large{2,9} && \Large{6,5} && \Large{5,1} & 
		\end{tabular} 
	\end{center}
  \item
	\begin{center}
		\begin{tabular}[h!]{l|ccccccc}
			&& \Large{$x_2$} && \Large{$y_2$} && \Large{$z_2$} & \\
			\hline
			\Large{$x_1$} && \Large{0,0} && \Large{5,4} && \Large{4,5} & \\
			\Large{$y_1$} && \Large{4,5} && \Large{0,0} && \Large{5,4} & \\
			\Large{$z_1$} && \Large{5,4} && \Large{4,5} && \Large{0,0} & 
		\end{tabular} 
	\end{center}
\end{enumerate}

\begin{solution}
    \begin{enumerate}
        \item
            To Find all Nash equilibria in the following games, we have to check all the support. Here we have 9 support where 4 are pure strategies. A strategy profile $\sigma$ is a Nash-Equilibrium on a support $D$ if and only if it satisfies the steps 1, 2 and 3 below :

\begin{procedure}
\begin{itemize}
\item For all player $i \in N$, for all pure strategy $c_k \in C_i$, define the variable 
$$ 0 \leq \sigma_i(c_k) \leq 1 \text{ (the probability that $i$ plays $c_k$)}.$$ 
\item For all choices of \emph{support} $D = (D_i)_{i \in N}$, $D_i \subseteq C_i$, 
\begin{enumerate}
\item \underline{Support}:
 For each player $i \in N$ and strategy $c \in C_i$, set
$$ \sigma_i(c) = 0 \text{ if $c \in C_i \backslash D_i$}, \text{and } \sigma_i(c) \geq 0 \text{ if $c \in D_i$},$$
with $ \sum_{c \in D_i} \sigma_i(c) = 1$. 
\label{chap3:prc2:support}
\item \underline{Payoffs}: For all player $i \in N$
$$ \forall d_i \in D_i:  \qquad  \sum_{c_{-i} \in D_{-i}} \left  ( \prod_{j \in N-i} \sigma_j(c_j) \right ) u_i(c_{-i}, d_i) = w_i.  $$
\label{chap3:prc2:equ}
\item \underline{Best Response}:  For all player $i \in N$
$$ \forall e_i \in C_i \backslash D_i:  \qquad \sum_{c_{-i} \in D_{-i}} \left  ( \prod_{j \in N-i} \sigma_j(c_j) \right ) u_i(c_{-i}, e_i) \leq w_i. $$
 \label{chap3:prc2:domi}
\end{enumerate}
\end{itemize}
\label{chap3:proc:computeNash}
\end{procedure}



\begin{itemize}
  \item[$\bullet$] Let's begin with pure strategy supports. 
\end{itemize}

- $\mathbf{\{x_1\}}\times\mathbf{\{x_2\}}$.

The payoff of the players are $w_1=2$ and $w_2=1$. It's easy to see that if the player 2 play $y_2$ instead of $x_2$ we have :
\begin{equation*}
    1=w_2<u_2(x_1,y_2)=2
\end{equation*}
This means that step 3 does not hold, so this pure strategy is not a equilibrium 

- $\mathbf{\{x_1\}}\times\mathbf{\{y_2\}}$.

The payoff of the players are $w_1=1$ and $w_2=2$. It's easy to see that if the player 1 play $y_1$ instead of $x_1$ we have :
\begin{equation*}
    1=w_1<u_1(y_1,y_2)=2
\end{equation*}
This means that step 3  equation does not hold, so this pure strategy is not a equilibrium 

- $\mathbf{\{y_1\}}\times\mathbf{\{x_2\}}$.

The payoff of the players are $w_1=1$ and $w_2=5$. It's easy to see that if the player 1 play $x_1$ instead of $y_1$ we have :
\begin{equation*}
    1=w_1<u_1(x_1,x_2)=2
\end{equation*}
This means that step 3  equation does not hold, so this pure strategy is not a equilibrium 

- $\mathbf{\{y_1\}}\times\mathbf{\{y_2\}}$.

The payoff of the players are $w_1=2$ and $w_2=1$. It's easy to see that if the player 2 play $x_2$ instead of $y_2$ we have :
\begin{equation*}
    1=w_2<u_2(y_1,x_2)=5
\end{equation*}
This means that step 3  equation does not hold, so this pure strategy is not a equilibrium 


\begin{itemize}
  \item[$\bullet$] Randomized vs pure strategy supports : This corresponds to the case when for example the first player randomizes between $x_1$ and $y_1$, while the second sticks either to $x_2$ or to $y_2$.
\end{itemize}

- $\mathbf{\{x_1,y_1\}}\times\mathbf{\{x_2\}}$.

By the Payoffs equation (step 2), we must have :
\begin{equation*}
    u_1(x_1,x_2)=u_1(y_1,x_2)=w_1
\end{equation*}
which cannot hold since $2>1$. Thus, the support does not lead to a Nash equilibrium.


-$\mathbf{\{x_1,y_1\}}\times\mathbf{\{y_2\}}$.

By the Payoffs equation (step 2), we must have :
 \begin{equation*}
    u_1(x_1,y_2)=u_1(y_1,y_2)=w_1
\end{equation*}
which cannot hold since $1<2$. Thus, the support does not lead to a Nash equilibrium.


- $\mathbf{\{x_1\}}\times\mathbf{\{x_2,y_2\}}$.

By the Payoffs equation (step 2), we must have :
 \begin{equation*}
    u_2(x_1,x_2)=u_2(x_1,y_2)=w_2
\end{equation*}
which cannot hold since $1<2$. Thus, the support does not lead to a Nash equilibrium.

- $\mathbf{\{y_1\}}\times\mathbf{\{x_2,y_2\}}$.

By the Payoffs equation (step 2), we must have :
 \begin{equation*}
    u_2(y_1,x_2)=u_2(y_1,y_2)=w_2
\end{equation*}
which cannot hold since $5>1$. Thus, the support does not lead to a Nash equilibrium.


\begin{itemize}
  \item[$\bullet$] Fully randomized strategy : $\mathbf{\{x_1,y_1\}}\times\mathbf{\{x_2,y_2\}}$.
\end{itemize}
Again we first check the second equation, for the player 1 we have : 
\begin{align*}
 \sigma_2(x_2)u_1(x_1,x_2) + \sigma_2(y_2)u_1(x_1,y_2) &= \sigma_2(x_2)u_1(y_1,x_2) + \sigma_2(y_2)u_1(y_1,y_2) \\
 2\sigma_2(x_2)+1\sigma_2(y_2)&= 1\sigma_2(x_2) + 2\sigma_2(y_2) \\
 \sigma_2(x_2)&=\sigma_2(y_2)
\end{align*}

Moreover we know that $\sigma_2(x_2) + \sigma_2(y_2) = 1$ and so we find $\sigma_2(x_2)=\frac{1}{2}$ and $\sigma_2(x_2)=\frac{1}{2}$

Now for the player 2 we have : 
\begin{align*}
 \sigma_1(x_1)u_2(x_1,x_2) + \sigma_1(y_1)u_2(y_1,x_2) &= \sigma_1(x_1)u_2(x_1,y_2) + \sigma_1(y_1)u_2(y_1,y_2) \\
 1\sigma_1(x_1)+5\sigma_1(y_1)&= 2\sigma_1(x_1) + 1\sigma_1(y_1) \\
 4\sigma_1(y_1)&=\sigma_1(x_1)
\end{align*}

We also know that $\sigma_1(x_1) + \sigma_1(y_1) = 1$ and so we find $\sigma_1(x_1)=\frac{4}{5}$ and $\sigma_1(y_1)=\frac{1}{5}$

To conclude we have one Nash equilibrium :
$\sigma = \left( \frac{4}{5}x_1 + \frac{1}{5}y_1, \frac{1}{2}x_2 + \frac{1}{2}y_2\right)$ and the payoff associated is $w= \left(\frac{3}{2},\frac{9}{5}\right)$
        

        \item
            We can proceed exactly as for the case~``\ref{item:first}''. We can first analyze the four pure strategies and we already find two Nash equilibria :

- $\mathbf{\{x_1\}}\times\mathbf{\{x_2\}}$. Indeed step 3 implies :
\begin{align*}
    3=w_1&\ge u_1(x_1,x_2)=2 \\
    7=w_2&\ge u_2(x_1,y_2)=6 
\end{align*}
And so all the step are respected (note that step 1 and 2 are trivial for the pure strategies) 

- $\mathbf{\{y_1\}}\times\mathbf{\{y_2\}}$. Indeed step 3 implies :
\begin{align*}
    7 =w_1&\ge u_1(x_1,y_2)=6 \\
    3=w_2&\ge u_2(y_1,x_2)=2
\end{align*}
Again all the step are respected.


\begin{itemize}
  \item[$\bullet$] Randomized vs pure strategy supports
\end{itemize}

One can develop Payoff equation (step 2) as in case~``\ref{item:first}'' and see that all Randomized vs pure strategy supports does not lead to a Nash equilibrium.

\begin{itemize}
  \item[$\bullet$] Fully randomized strategy 
\end{itemize}
- $\mathbf{\{x_1,y_1\}}\times\mathbf{\{x_2,y_2\}}$.

Again we first check the payoff equations (step 2), for the player 1 ($i=1$) we have : 
\begin{align*}
 \sigma_2(x_2)u_1(x_1,x_2) + \sigma_2(y_2)u_1(x_1,y_2) &= \sigma_2(x_2)u_1(y_1,x_2) + \sigma_2(y_2)u_1(y_1,y_2) \\
 3\sigma_2(x_2)+6\sigma_2(y_2)&= 2\sigma_2(x_2) + 7\sigma_2(y_2) \\
 \sigma_2(x_2)&=\sigma_2(y_2)
\end{align*}

Moreover we know that $\sigma_2(x_2) + \sigma_2(y_2) = 1$ and so we find $\sigma_2(x_2)=\frac{1}{2}$ and $\sigma_2(x_2)=\frac{1}{2}$

Now for the player 2 ($i=2$) we have : 
\begin{align*}
 \sigma_1(x_1)u_2(x_1,x_2) + \sigma_1(y_1)u_2(y_1,x_2) &= \sigma_1(x_1)u_2(x_1,y_2) + \sigma_1(y_1)u_2(y_1,y_2) \\
 7\sigma_1(x_1)+2\sigma_1(y_1)&= 6\sigma_1(x_1) + 3\sigma_1(y_1) \\
 \sigma_1(x_1)&=\sigma_1(y_1)
\end{align*}

We also know that $\sigma_1(x_1) + \sigma_1(y_1) = 1$ and so we find $\sigma_1(x_1)=\frac{1}{2}$ and $\sigma_1(y_1)=\frac{1}{2}$

To conclude we have 3 Nash equilibria :
\begin{itemize}
  \item[$\bullet$]  $\sigma^1 = \left(x_1,x_2\right)$ with $w^1= \left(3,7\right)$
  \item[$\bullet$]  $\sigma^2 = \left(y_1,y_2\right)$ with $w^2= \left(7,3\right)$
  \item[$\bullet$]  $\sigma^3 = \left( \frac{1}{2}x_1 + \frac{1}{2}y_1, \frac{1}{2}x_2 + \frac{1}{2}y_2\right)$ with $w^3= \left(\frac{9}{2},\frac{9}{2}\right)$
\end{itemize}

        \item
            We see that strategy $x_{2}$ is strongly dominated by the pure strategy strategy $y_{2}$. Here we look at the columns, hence we are looking at player 2, hence we have to consider the payoffs of player 2. The strong domination is clear, since $3 < 6$ and $2 < 7$. The table becomes

\begin{tabular}[h!]{l|ccccc}
		&& $y_2$ \\
		\hline
		$x_1$ && 6,6 & \\
		$y_1$ && 3,7 &
	\end{tabular} 
  
  \vspace{5mm}
  
We see that strategy $y_{1}$ is strongly dominated by the pure strategy strategy $x_{1}$. Here we look at the rows, hence we are looking at player 1, hence we have to consider the payoffs of player 1. The strong domination is clear, since $3 < 6$. The table becomes

\begin{tabular}[h!]{l|ccccc}
		&& $y_2$ \\
		\hline
		$x_1$ && 6,6 &
\end{tabular} 

\vspace{5mm}

Hence the only equilibrium of this game is $\parent{x_1, y_2}$, and the associated payoff is $w = \parent{6,6}$.
        \item
            Unfortunately we don't have strongly dominated strategies in this game so we have a lot of supports to check. Let's begin with pure strategies. We can find 2 Nash equilibria :

- $\mathbf{\{x_1\}}\times\mathbf{\{z_2\}}$. Indeed step 3 is verified :
\begin{align*}
    8=w_1&\ge u_1(y_1,z_2)=5   \\
    7=w_2&\ge u_2(x_1,y_2)=5 \quad \mbox{and} \quad 7=w_2\ge u_2(x_1,x_2)=4 
\end{align*}


- $\mathbf{\{y_1\}}\times\mathbf{\{x_2\}}$. Indeed step 3 is verified :
\begin{align*}
    2=w_1&\ge u_1(x_1,x_2)=0 \\
    9=w_2&\ge u_2(y_1,y_2)=5  \quad \mbox{and} \quad  9=w_2\ge u_2(y_1,z_2)=1
\end{align*}

\begin{itemize}
  \item[$\bullet$] Randomized vs pure strategy supports
\end{itemize}

Fortunately we can quickly check these support with the payoff equations (step 2). We can also notice that for example if the support $\mathbf{\{x_1\}}\times\mathbf{\{x_2,y_2\}}$  does not lead to a Nash equilibrium, the support $\mathbf{\{x_1\}}\times\mathbf{\{x_2,y_2,z_2\}}$ does not lead to a Nash equilibrium either. 

- $\mathbf{\{x_1,y1\}}\times\mathbf{\{x_2\}}$, $\mathbf{\{x_1,y1\}}\times\mathbf{\{y_2\}}$ and $\mathbf{\{x_1,y1\}}\times\mathbf{\{z_2\}}$ does not lead to a Nash equilibrium , indeed :
\begin{equation*}
    0\ne2 \quad 5\ne6 \quad 8\ne5
\end{equation*}

- $\mathbf{\{x_1\}}\times\mathbf{\{x_2,y_2\}}$, $\mathbf{\{x_1\}}\times\mathbf{\{x_2,z_2\}}$ and $\mathbf{\{x_1\}}\times\mathbf{\{y_2,z_2\}}$ does not lead to a Nash equilibrium , indeed :
\begin{equation*}
    4\ne6 \quad 4\ne7 \quad 6\ne7
\end{equation*}

- $\mathbf{\{y_1\}}\times\mathbf{\{x_2,y_2\}}$, $\mathbf{\{y_1\}}\times\mathbf{\{x_2,z_2\}}$ and $\mathbf{\{y_1\}}\times\mathbf{\{y_2,z_2\}}$ does not lead to a Nash equilibrium , indeed :
\begin{equation*}
    9\ne5 \quad 9\ne1 \quad 5\ne1
\end{equation*}


And so as explained $\mathbf{\{x_1\}}\times\mathbf{\{x_2,y_2,z_2\}}$ and $\mathbf{\{y_1\}}\times\mathbf{\{x_2,y_2,z_2\}}$ do not lead to Nash equilibria.


\begin{itemize}
  \item[$\bullet$] Fully randomized strategy : two vs two
\end{itemize}

Here we have 3 support to check, first we have :

-$\mathbf{\{x_1,y_1\}}\times\mathbf{\{x_2,y_2\}}$

 We can begin to compute payoff equation (step 2) for player 1 ($i=1$) :
 \begin{align*}
 \sigma_2(x_2)u_1(x_1,x_2) + \sigma_2(y_2)u_1(x_1,y_2) &= \sigma_2(x_2)u_1(y_1,x_2) + \sigma_2(y_2)u_1(y_1,y_2) \\
 0\sigma_2(x_2)+5\sigma_2(y_2)&= 2\sigma_2(x_2) + 6\sigma_2(y_2) \\
 2\sigma_2(x_2)&=-\sigma_2(y_2)
\end{align*}

We know that probabilities are always positive so the equation does not hold here. we can stop


-$\mathbf{\{x_1,y_1\}}\times\mathbf{\{x_2,z_2\}}$

Again we first check the payoff equations (step 2), for the player 1 ($i=1$) we have : 
\begin{align*}
 \sigma_2(x_2)u_1(x_1,x_2) + \sigma_2(z_2)u_1(x_1,z_2) &= \sigma_2(x_2)u_1(y_1,x_2) + \sigma_2(z_2)u_1(y_1,z_2) \\
 0\sigma_2(x_2)+8\sigma_2(z_2)&= 2\sigma_2(x_2) + 5\sigma_2(z_2) \\
 2\sigma_2(x_2)&=3\sigma_2(z_2)
\end{align*}

Moreover we know that $\sigma_2(x_2) + \sigma_2(z_2) = 1$ and so we find $\sigma_2(x_2)=\frac{3}{5}$ and $\sigma_2(z_2)=\frac{2}{5}$

Now for the player 2 ($i=2$) we have : 
\begin{align*}
 \sigma_1(x_1)u_2(x_1,x_2) + \sigma_1(y_1)u_2(y_1,x_2) &= \sigma_1(x_1)u_2(x_1,z_2) + \sigma_1(y_1)u_2(y_1,z_2) \\
 4\sigma_1(x_1)+9\sigma_1(y_1)&= 7\sigma_1(x_1) + 1\sigma_1(y_1) \\
 3\sigma_1(x_1)&=8\sigma_1(y_1)
\end{align*}

We also know that $\sigma_1(x_1) + \sigma_1(y_1) = 1$ and so we find $\sigma_1(x_1)=\frac{8}{11}$ and $\sigma_1(y_1)=\frac{3}{11}$

WARNING : Here we have to check the step 3 ! (Is this the best response ?) For the player 2 ($i=2$), $e_2=y_2$ with $w_2=\frac{59}{11}$ we obtain :
\begin{equation*}
    \frac{8}{11}6 + \frac{3}{11}5 \le w_2
\end{equation*}
And so this support does not lead to a Nash equilibrium because $\frac{63}{11}>\frac{59}{11}$

-$\mathbf{\{x_1,y_1\}}\times\mathbf{\{y_2,z_2\}}$

Using payoff equations (step 2), for the player 1 ($i=1$) we have : 
\begin{align*}
 \sigma_2(y_2)u_1(x_1,y_2) + \sigma_2(z_2)u_1(x_1,z_2) &= \sigma_2(y_2)u_1(y_1,y_2) + \sigma_2(z_2)u_1(y_1,z_2) \\
 5\sigma_2(y_2)+8\sigma_2(z_2)&= 6\sigma_2(y_2) + 5\sigma_2(z_2) \\
 \sigma_2(y_2)&=3\sigma_2(z_2)
\end{align*}

Moreover we know that $\sigma_2(y_2) + \sigma_2(z_2) = 1$ and so we find $\sigma_2(y_2)=\frac{3}{4}$ and $\sigma_2(z_2)=\frac{1}{4}$

Now for the player 2 ($i=2$) we have : 
\begin{align*}
 \sigma_1(x_1)u_2(x_1,y_2) + \sigma_1(y_1)u_2(y_1,y_2) &= \sigma_1(x_1)u_2(x_1,z_2) + \sigma_1(y_1)u_2(y_1,z_2) \\
 6\sigma_1(x_1)+5\sigma_1(y_1)&= 7\sigma_1(x_1) + 1\sigma_1(y_1) \\
 \sigma_1(x_1)&=4\sigma_1(y_1)
\end{align*}

We also know that $\sigma_1(x_1) + \sigma_1(y_1) = 1$ and so we find $\sigma_1(x_1)=\frac{4}{5}$ and $\sigma_1(y_1)=\frac{1}{5}$

Again we have to check the best response equation (step 3). for the player 2 ($i=2$) $e_2=x_2$ with $w_2=\frac{29}{5}$ we obtain :
\begin{equation*}
    \frac{4}{5}4 + \frac{1}{5}9 \le w_2
\end{equation*}
And this support lead to a Nash equilibrium because $\frac{29}{5}>\frac{25}{5}$



\begin{itemize}
  \item[$\bullet$] Fully randomized strategy 
\end{itemize}

Finally we have the support $\mathbf{\{x_1,y_1\}}\times\mathbf{\{x_2,y_2,z_2\}}$. It's not useful to recompute everything, we can directly see that this support does not lead to a Nash equilibrium. Indeed the payoff equations (step 2) for the player 2 does not hold because the payoff equations for the support $\mathbf{\{x_1,y_1\}}\times\mathbf{\{x_2,y_2\}}$  does not hold.


To conclude we have 3 Nash equilibria :
\begin{itemize}
  \item[$\bullet$]  $\sigma^1 = \left(x_1,z_2\right)$ with $w^1= \left(8,7\right)$
  \item[$\bullet$]  $\sigma^2 = \left(y_1,x_2\right)$ with $w^2= \left(2,9\right)$
  \item[$\bullet$]  $\sigma^3 = \left( \frac{4}{5}x_1 + \frac{1}{5}y_1, \frac{3}{4}y_2 + \frac{1}{4}z_2\right)$ with $w^3= \left(\frac{23}{4},\frac{29}{5}\right)$
\end{itemize}

        \item
            \paragraph{Domination in subgames}
While not mandatory to solve this exercise, it can drastically simplify it to
note that
\begin{itemize}
  \item in the subgame $\{x_1, z_1\} \times \{x_2, y_2, z_2\}$,
    the strategy $x_2$ is dominated so $x_2$ cannot be in a nash equilibrium
    for which the support of the first player is a subset of $\{x_1, z_1\}$;
  \item in the subgame $\{x_1, y_1\} \times \{x_2, y_2, z_2\}$,
    the strategy $y_2$ is dominated so $y_2$ cannot be in a nash equilibrium
    for which the support of the first player is a subset of $\{x_1, y_1\}$;
  \item in the subgame $\{y_1, z_1\} \times \{x_2, y_2, z_2\}$,
    the strategy $z_2$ is dominated so $z_2$ cannot be in a nash equilibrium
    for which the support of the first player is a subset of $\{y_1, z_1\}$;
  \item in the subgame $\{x_1, y_1, z_1\} \times \{x_2, z_2\}$,
    the strategy $x_1$ is dominated so $x_1$ cannot be in a nash equilibrium
    for which the support of the first player is a subset of $\{x_2, z_2\}$;
  \item in the subgame $\{x_1, y_1, z_1\} \times \{x_2, y_2\}$,
    the strategy $y_1$ is dominated so $y_1$ cannot be in a nash equilibrium
    for which the support of the first player is a subset of $\{x_2, y_2\}$;
  \item in the subgame $\{x_1, y_1, z_1\} \times \{y_2, z_2\}$,
    the strategy $z_1$ is dominated so $z_1$ cannot be in a nash equilibrium
    for which the support of the first player is a subset of $\{y_2, z_2\}$.
\end{itemize}
Therefore, the only support to check is $\{x_1, y_1, z_1\} \times \{x_2, y_2, z_2\}$.
The solution below also analyzes the other cases and find no equilibrium as we
have just deduced.

\begin{enumerate} [label=\Alph*. ]
    \item \textbf{Pure strategy VS. Pure strategy.} \\
Let us begin with pure strategies only. We see that we cannot find any Nash equilibrium. Let us see that rigorously. We have to check nine supports.

\begin{enumerate} [label*= (\arabic*)]
	\item $\mathbf{\{x_1\}} \times \mathbf{\{x_2\}}$. \\
	      In Step 2, the payoffs for each player are $w_1 = 0$ and $w_2 = 0$. It is easy to see that if the first player played $z_1$ instead of $x_1$, we have
	      \begin{equation*}
	      	0 = w_1 < u_1 \parent{z_1, x_2} = 5.
	      \end{equation*}
	      This means that Step 3 does not hold, so $\mathbf{\{x_1\}} \times \mathbf{\{x_2\}}$ is not a equilibrium.
	      
	\item $\mathbf{\{y_1\}} \times \mathbf{\{y_2\}}$. \\
	      In Step 2, the payoffs for each player are $w_1 = 0$ and $w_2 = 0$. It is easy to see that if the first player played $x_1$ instead of $y_1$, we have
	      \begin{equation*}
	      	0 = w_1 < u_1 \parent{x_1, y_2} = 5.
	      \end{equation*}
	      This means that Step 3 does not hold, so $\mathbf{\{y_1\}} \times \mathbf{\{y_2\}}$ is not a equilibrium.
	      
	\item $\mathbf{\{z_1\}} \times \mathbf{\{z_2\}}$. \\
	      In Step 2, the payoffs for each player are $w_1 = 0$ and $w_2 = 0$. It is easy to see that if the first player played $y_1$ instead of $z_1$, we have
	      \begin{equation*}
	      	0 = w_1 < u_1 \parent{y_1, z_2} = 5.
	      \end{equation*}
	      This means that Step 3 does not hold, so $\mathbf{\{z_1\}} \times \mathbf{\{z_2\}}$ is not a equilibrium.
	          
	          
	          
	\item $\mathbf{\{x_1\}} \times \mathbf{\{y_2\}}$. \\
	      In Step 2, the payoffs for each player are $w_1 = 5$ and $w_2 = 4$. It is easy to see that if the second player played $z_2$ instead of $y_2$, we have
	      \begin{equation*}
	      	4 = w_2 < u_2 \parent{x_1, z_2} = 5.
	      \end{equation*}
	      This means that Step 3 does not hold, so $\mathbf{\{x_1\}} \times \mathbf{\{y_2\}}$ is not a equilibrium.
	          
	\item $\mathbf{\{x_1\}} \times \mathbf{\{z_2\}}$. \\
	      In Step 2, the payoffs for each player are $w_1 = 4$ and $w_2 = 5$. It is easy to see that if the first player played $y_1$ instead of $x_1$, we have
	      \begin{equation*}
	      	4 = w_1 < u_1 \parent{y_1, z_2} = 5.
	      \end{equation*}
	      This means that Step 3 does not hold, so $\mathbf{\{x_1\}} \times \mathbf{\{z_2\}}$ is not a equilibrium.
	      
	\item $\mathbf{\{y_1\}} \times \mathbf{\{z_2\}}$. \\
	      In Step 2, the payoffs for each player are $w_1 = 5$ and $w_2 = 4$. It is easy to see that if the second player played $x_2$ instead of $z_2$, we have
	      \begin{equation*}
	      	4 = w_2 < u_2 \parent{y_1, x_2} = 5.
	      \end{equation*}
	      This means that Step 3 does not hold, so $\mathbf{\{y_1\}} \times \mathbf{\{z_2\}}$ is not a equilibrium.
	      
	\vspace{3mm}
	
	\item[] Now we can use the symmetry of the problem.
	
	\item The support $\mathbf{\{x_1\}} \times \mathbf{\{y_2\}}$ does not give an equilibrium, therefore the support $\mathbf{\{y_1\}} \times \mathbf{\{x_2\}}$ will not give an equilibrium either.
	
	\item The support $\mathbf{\{x_1\}} \times \mathbf{\{z_2\}}$ does not give an equilibrium, therefore the support $\mathbf{\{z_1\}} \times \mathbf{\{x_2\}}$ will not give an equilibrium either.
	
	\item The support $\mathbf{\{y_1\}} \times \mathbf{\{z_2\}}$ does not give an equilibrium, therefore the support $\mathbf{\{z_1\}} \times \mathbf{\{y_2\}}$ will not give an equilibrium either.
\end{enumerate}

\item \textbf{Pure strategy VS. Randomized strategy with two choices.} \\
Here we look at support where player 1 has a pure strategy and player 2 has a randomized strategy with two choices. Again, we see that we cannot find any Nash equilibrium. Let us see that rigorously. We have nine supports to check.

\begin{enumerate} [label*= (\arabic*)]
    \item $\mathbf{\{x_1\}} \times \mathbf{\{x_2, y_2\}}$. \\
        In Step 2, for player 2, we have
        \begin{equation*}
            w_2 = u_2 \parent{x_1, d_2} \ \forall \ d_2 \in \bracket{x_2, y_2}.
        \end{equation*}
        Hence we must have 0 = 4, therefore, this support does not give an equilibrium.
        
    \item $\mathbf{\{x_1\}} \times \mathbf{\{x_2, z_2\}}$. \\
        In Step 2, for player 2, we have
        \begin{equation*}
            w_2 = u_2 \parent{x_1, d_2} \ \forall \ d_2 \in \bracket{x_2, z_2}.
        \end{equation*}
        Hence we must have 0 = 5, therefore, this support does not give an equilibrium.
        
    \item $\mathbf{\{x_1\}} \times \mathbf{\{y_2, z_2\}}$. \\
        In Step 2, for player 2, we have
        \begin{equation*}
            w_2 = u_2 \parent{x_1, d_2} \ \forall \ d_2 \in \bracket{y_2, z_2}.
        \end{equation*}
        Hence we must have 4 = 5, therefore, this support does not give an equilibrium.
        
    \item $\mathbf{\{y_1\}} \times \mathbf{\{x_2, y_2\}}$. \\
        In Step 2, for player 2, we have
        \begin{equation*}
            w_2 = u_2 \parent{y_1, d_2} \ \forall \ d_2 \in \bracket{x_2, y_2}.
        \end{equation*}
        Hence we must have 5 = 0, therefore, this support does not give an equilibrium.
        
    \item $\mathbf{\{y_1\}} \times \mathbf{\{x_2, z_2\}}$. \\
        In Step 2, for player 2, we have
        \begin{equation*}
            w_2 = u_2 \parent{y_1, d_2} \ \forall \ d_2 \in \bracket{x_2, z_2}.
        \end{equation*}
        Hence we must have 5 = 4, therefore, this support does not give an equilibrium.
        
    \item $\mathbf{\{y_1\}} \times \mathbf{\{y_2, z_2\}}$. \\
        In Step 2, for player 2, we have
        \begin{equation*}
            w_2 = u_2 \parent{y_1, d_2} \ \forall \ d_2 \in \bracket{y_2, z_2}.
        \end{equation*}
        Hence we must have 0 = 4, therefore, this support does not give an equilibrium.
        
    \item $\mathbf{\{z_1\}} \times \mathbf{\{x_2, y_2\}}$. \\
        In Step 2, for player 2, we have
        \begin{equation*}
            w_2 = u_2 \parent{z_1, d_2} \ \forall \ d_2 \in \bracket{x_2, y_2}.
        \end{equation*}
        Hence we must have 4 = 5, therefore, this support does not give an equilibrium.
        
    \item $\mathbf{\{z_1\}} \times \mathbf{\{x_2, z_2\}}$. \\
        In Step 2, for player 2, we have
        \begin{equation*}
            w_2 = u_2 \parent{z_1, d_2} \ \forall \ d_2 \in \bracket{x_2, z_2}.
        \end{equation*}
        Hence we must have 4 = 0, therefore, this support does not give an equilibrium.
        
    \item $\mathbf{\{z_1\}} \times \mathbf{\{y_2, z_2\}}$. \\
        In Step 2, for player 2, we have
        \begin{equation*}
            w_2 = u_2 \parent{z_1, d_2} \ \forall \ d_2 \in \bracket{y_2, z_2}.
        \end{equation*}
        Hence we must have 5 = 0, therefore, this support does not give an equilibrium.
\end{enumerate}



\item \textbf{Randomized strategy with two choices VS. Pure strategy.} \\
Here we look at support where player 1 has a randomized strategy with two choices, and player 2 has a pure strategy. We have nine supports to check. Using the symmetry of the game, we can show that the supports
\begin{enumerate} [label*= (\arabic*)]
    \item $\mathbf{\{x_1, y_1\}} \times \mathbf{\{x_2\}}$
    \item $\mathbf{\{x_1, z_1\}} \times \mathbf{\{x_2\}}$
    \item $\mathbf{\{y_1, z_1\}} \times \mathbf{\{x_2\}}$
    
    \item $\mathbf{\{x_1, y_1\}} \times \mathbf{\{y_2\}}$
    \item $\mathbf{\{x_1, z_1\}} \times \mathbf{\{y_2\}}$
    \item $\mathbf{\{y_1, z_1\}} \times \mathbf{\{y_2\}}$
    
    \item $\mathbf{\{x_1, y_1\}} \times \mathbf{\{z_2\}}$
    \item $\mathbf{\{x_1, z_1\}} \times \mathbf{\{z_2\}}$
    \item $\mathbf{\{y_1, z_1\}} \times \mathbf{\{z_2\}}$
\end{enumerate}

do not give an equilibrium.


\item \textbf{Pure strategy VS. Randomized strategy with three choices.} \\
Here we look at support where player 1 has a pure strategy and player 2 has a randomized strategy with three choices. We see that we cannot find any Nash equilibrium. Let us see that rigorously. We have three supports to check.

\begin{enumerate} [label*= (\arabic*)]
    \item The support $\mathbf{\{x_1\}} \times \mathbf{\{x_2, y_2, z_2\}}$ does not give an equilibrium because the support $\mathbf{\{x_1\}} \times \mathbf{\{x_2, y_2\}}$ does not give an equilibrium.
    
    \item The support $\mathbf{\{y_1\}} \times \mathbf{\{x_2, y_2, z_2\}}$ does not give an equilibrium because the support $\mathbf{\{y_1\}} \times \mathbf{\{x_2, y_2\}}$ does not give an equilibrium.
    
    \item The support $\mathbf{\{z_1\}} \times \mathbf{\{x_2, y_2, z_2\}}$ does not give an equilibrium because the support $\mathbf{\{z_1\}} \times \mathbf{\{x_2, y_2\}}$ does not give an equilibrium.
\end{enumerate}

\item \textbf{Randomized strategy with three choices VS. Pure strategy.} \\
Here we look at support where player 1 has a randomized strategy with three choices, and player 2 has a pure strategy. We have three supports to check. Using the symmetry of the game, we can show that the supports
\begin{enumerate} [label*= (\arabic*)]
    \item $\mathbf{\{x_1, y_1, z_1\}} \times \mathbf{\{x_2\}}$
    \item $\mathbf{\{x_1, y_1, z_1\}} \times \mathbf{\{y_2\}}$
    \item $\mathbf{\{x_1, y_1, z_1\}} \times \mathbf{\{z_2\}}$
\end{enumerate}

do not give an equilibrium.

\item \textbf{Randomized strategy with two choices VS. Randomized strategy with two choices.} \\
Here we look at support where players 1 and 2 both have a randomized strategy with two choices. We see that we cannot find any Nash equilibrium. Let us see that rigorously. We have nine supports to check.

\begin{enumerate} [label*= (\arabic*)]
    \item $\mathbf{\{x_1, y_1\}} \times \mathbf{\{x_2, y_2\}}$. \\
    We define $\sigma_{1} \parent{x_1} = \alpha$ and $\sigma_{1} \parent{y_1} = 1 - \alpha$. In Step 2, for player 2, we have
    \begin{equation*}
        w_2 = \sum_{c_1 \in \bracket{x_1, y_1}} \sigma_{1} \parent{c_1} \cdot u_2 \parent{c_1, d_2} \ \forall \ d_2 \in \bracket{x_2, y_2}.
    \end{equation*}
    Hence we must have $w_2 = 0 \cdot \alpha + 5 \cdot \parent{1 - \alpha} = 4 \cdot \alpha + 0 \cdot \parent{1 - \alpha}$, or again, $\alpha = \frac{5}{9}$. Then we obtain $w_2 = \frac{20}{9}$. We see that Step 3 is not respected because $e_2 = z_2 \in C_2 \backslash D_2$ is such that 
    \begin{equation*}
        \sum_{c_{1} \in D_{1}} \sigma_{1} \parent{c_1} u_2 \parent{c_{1}, e_2}
        = \alpha \cdot 5 + \parent{1 - \alpha} \cdot 4
        = \dfrac{45 + 16}{9}
        = \dfrac{51}{9}
        > \dfrac{20}{9}
        = w_2.
    \end{equation*}
    Hence this support does not yield an equilibrium.

    
    \item $\mathbf{\{x_1, z_1\}} \times \mathbf{\{x_2, z_2\}}$. \\
    We define $\sigma_{1} \parent{x_1} = \alpha$ and $\sigma_{1} \parent{z_1} = 1 - \alpha$. In Step 2, for player 2, we have
    \begin{equation*}
        w_2 = \sum_{c_1 \in \bracket{x_1, z_1}} \sigma_{1} \parent{c_1} \cdot u_2 \parent{c_1, d_2} \ \forall \ d_2 \in \bracket{x_2, z_2}.
    \end{equation*}
    Hence we must have $w_2 = 0 \cdot \alpha + 4 \cdot \parent{1 - \alpha} = 5 \cdot \alpha + 0 \cdot \parent{1 - \alpha}$, or again, $\alpha = \frac{4}{9}$. Then we obtain $w_2 = \frac{20}{9}$. We see that Step 3 is not respected because $e_2 = y_2 \in C_2 \backslash D_2$ is such that 
    \begin{equation*}
        \sum_{c_{1} \in D_{1}} \sigma_{1} \parent{c_1} u_2 \parent{c_{1}, e_2}
        = \alpha \cdot 4 + \parent{1 - \alpha} \cdot 5
        = \dfrac{16 + 45}{9}
        = \dfrac{51}{9}
        > \dfrac{20}{9}
        = w_2.
    \end{equation*}
    Hence this support does not yield an equilibrium.
    
    
    
    \item $\mathbf{\{y_1, z_1\}} \times \mathbf{\{y_2, z_2\}}$. \\
    We define $\sigma_{1} \parent{y_1} = \alpha$ and $\sigma_{1} \parent{z_1} = 1 - \alpha$. In Step 2, for player 2, we have
    \begin{equation*}
        w_2 = \sum_{c_1 \in \bracket{y_1, z_1}} \sigma_{1} \parent{c_1} \cdot u_2 \parent{c_1, d_2} \ \forall \ d_2 \in \bracket{y_2, z_2}.
    \end{equation*}
    Hence we must have $w_2 = 0 \cdot \alpha + 5 \cdot \parent{1 - \alpha} = 4 \cdot \alpha + 0 \cdot \parent{1 - \alpha}$, or again, $\alpha = \frac{5}{9}$. Then we obtain $w_2 = \frac{20}{9}$. We see that Step 3 is not respected because $e_2 = x_2 \in C_2 \backslash D_2$ is such that 
    \begin{equation*}
        \sum_{c_{1} \in D_{1}} \sigma_{1} \parent{c_1} u_2 \parent{c_{1}, e_2}
        = \alpha \cdot 5 + \parent{1 - \alpha} \cdot 4
        = \dfrac{45 + 16}{9}
        = \dfrac{51}{9}
        > \dfrac{20}{9}
        = w_2.
    \end{equation*}
    Hence this support does not yield an equilibrium.
    
    
    
    
    
    
    \item $\mathbf{\{x_1, y_1\}} \times \mathbf{\{x_2, z_2\}}$. \\
    We define $\sigma_{2} \parent{x_2} = \beta$ and $\sigma_{2} \parent{z_2} = 1 - \beta$. In Step 2, for player 1, we have
    \begin{equation*}
        w_1 = \sum_{c_2 \in \bracket{x_2, z_2}} \sigma_{2} \parent{c_2} \cdot u_1 \parent{c_2, d_1} \ \forall \ d_1 \in \bracket{x_1, y_1}.
    \end{equation*}
    Hence we must have $w_1 = 0 \cdot \beta + 4 \cdot \parent{1 - \beta} = 4 \cdot \beta + 5 \cdot \parent{1 - \beta}$, or again, $\beta = -\frac{1}{3}$. It is absurd to have a strictly negative probability, hence this support does not yield an equilibrium.
        
    
    \item $\mathbf{\{x_1, y_1\}} \times \mathbf{\{y_2, z_2\}}$. \\
    We define $\sigma_{1} \parent{x_1} = \alpha$ and $\sigma_{1} \parent{y_1} = 1 - \alpha$. In Step 2, for player 2, we have
    \begin{equation*}
        w_2 = \sum_{c_1 \in \bracket{x_1, y_1}} \sigma_{1} \parent{c_1} \cdot u_2 \parent{c_1, d_2} \ \forall \ d_2 \in \bracket{y_2, z_2}.
    \end{equation*}
    Hence we must have $w_2 = 4 \cdot \alpha + 0 \cdot \parent{1 - \alpha} = 5 \cdot \alpha + 4 \cdot \parent{1 - \alpha}$, or again, $\alpha = \frac{4}{3}$. It is absurd to have a probability strictly greater than one, hence this support does not yield an equilibrium.
    
    \item $\mathbf{\{x_1, z_1\}} \times \mathbf{\{y_2, z_2\}}$. \\
    We define $\sigma_{2} \parent{y_2} = \beta$ and $\sigma_{1} \parent{z_1} = 1 - \beta$. In Step 2, for player 1, we have
    \begin{equation*}
        w_1 = \sum_{c_2 \in \bracket{y_2, z_2}} \sigma_{2} \parent{c_2} \cdot u_1 \parent{c_2, d_1} \ \forall \ d_1 \in \bracket{x_1, z_1}.
    \end{equation*}
    Hence we must have $w_1 = 5 \cdot \beta + 4 \cdot \parent{1 - \beta} = 4 \cdot \beta + 0 \cdot \parent{1 - \beta}$, or again, $\beta = \frac{4}{3}$. It is absurd to have a probability strictly greater than one, hence this support does not yield an equilibrium.
    
    \vspace{3mm}
    
    
    
    \item[] Now we can use the symmetry of the problem.
    
    \item The support $\mathbf{\{x_1, y_1\}} \times \mathbf{\{x_2, z_2\}}$ does not give an equilibrium, therefore the support $\mathbf{\{x_1, z_1\}} \times \mathbf{\{x_2, y_2\}}$ will not give an equilibrium either.
    
    \item The support $\mathbf{\{x_1, y_1\}} \times \mathbf{\{y_2, z_2\}}$ does not give an equilibrium, therefore the support $\mathbf{\{y_1, z_1\}} \times \mathbf{\{x_2, y_2\}}$ will not give an equilibrium either.
    
    \item The support $\mathbf{\{x_1, z_1\}} \times \mathbf{\{y_2, z_2\}}$ does not give an equilibrium, therefore the support $\mathbf{\{y_1, z_1\}} \times \mathbf{\{x_2, z_2\}}$ will not give an equilibrium either.
        
\end{enumerate}

\item \textbf{Randomized strategy with two choices VS. Randomized strategy with three choices.} \\
Here we look at support where player 1 has a randomized strategy with two choices, and player 2 has a randomized strategy with three choices. We see that we cannot find any Nash equilibrium. Let us see that rigorously. We have three supports to check.
\begin{enumerate} [label*= (\arabic*)]
    \item The support $\mathbf{\{x_1, y_1\}} \times \mathbf{\{x_2, y_2, z_2\}}$ does not give an equilibrium because the support $\mathbf{\{x_1, y_1\}} \times \mathbf{\{x_2, y_2\}}$ does not give an equilibrium.
    
    \item The support $\mathbf{\{x_1, z_1\}} \times \mathbf{\{x_2, y_2, z_2\}}$ does not give an equilibrium because the support $\mathbf{\{x_1, z_1\}} \times \mathbf{\{x_2, y_2\}}$ does not give an equilibrium.
    
    \item The support $\mathbf{\{y_1, z_1\}} \times \mathbf{\{x_2, y_2, z_2\}}$ does not give an equilibrium because the support $\mathbf{\{y_1, z_1\}} \times \mathbf{\{x_2, y_2\}}$ does not give an equilibrium.
\end{enumerate}

\item \textbf{Randomized strategy with three choices VS. Randomized strategy with two choices.} \\
Here we look at support where player 1 has a randomized strategy with three choices, and player 2 has a randomized strategy with two choices. We have three supports to check. Using the symmetry of the game, we can show that the supports
\begin{enumerate} [label*= (\arabic*)]
    \item $\mathbf{\{x_1, y_1, z_1\}} \times \mathbf{\{x_2, y_2\}}$
    \item $\mathbf{\{x_1, y_1, z_1\}} \times \mathbf{\{x_2, z_2\}}$
    \item $\mathbf{\{x_1, y_1, z_1\}} \times \mathbf{\{y_2, z_2\}}$
\end{enumerate}

do not give an equilibrium.

\item \textbf{Randomized strategy with three choices VS. Randomized strategy with three choices.} \\
This is the last paragraph of this long exercise! Hopefully there are not too many typos so far! What is sure is that we haven't found any Nash equilibrium so far. We know that there must be at least one Nash equilibrium in any game, therefore we know for sure that this combination of support will give a Nash equilibrium! Let us check that. We have only one support to check, i.e., the support $\mathbf{\{x_1, y_1, z_1\}} \times \mathbf{\{x_2, y_2, z_2\}}$.
One the one hand, we define $\sigma_{1} \parent{x_1} = \alpha_1$, $\sigma_{1} \parent{y_1} = \beta_1$ and $\sigma_{1} \parent{z_1} = \gamma_1$.
One the other hand, we define $\sigma_{2} \parent{x_2} = \alpha_2$, $\sigma_{2} \parent{y_2} = \beta_2$ and $\sigma_{2} \parent{z_2} = \gamma_2$.

For player 1, Step 2 gives
\begin{equation*}
    w_1 = \sum_{c_2 \in \bracket{x_2, y_2, z_2}} \sigma_{2} \parent{c_2} \cdot u_1 \parent{c_2, d_1} \ \forall \ d_1 \in \bracket{x_1, y_1, z_1}.
\end{equation*}
Hence we must have $w_1 = 0 \cdot \alpha_2 + 5 \cdot \beta_2 + 4 \cdot \gamma_2
= 4 \cdot \alpha_2 + 0 \cdot \beta_2 + 5 \cdot \gamma_2
= 5 \cdot \alpha_2 + 4 \cdot \beta_2 + 0 \cdot \gamma_2$. We have three equations and three unknowns. We find $\alpha_2 = \beta_2 = \gamma_2 = \frac{1}{3}$.

For player 2, Step 2 gives
\begin{equation*}
    w_2 = \sum_{c_1 \in \bracket{x_1, y_1, z_1}} \sigma_{1} \parent{c_1} \cdot u_2 \parent{c_1, d_2} \ \forall \ d_2 \in \bracket{x_2, y_2, z_2}.
\end{equation*}
Hence we must have $w_2 = 0 \cdot \alpha_1 + 5 \cdot \beta_1 + 4 \cdot \gamma_1
= 4 \cdot \alpha_1 + 0 \cdot \beta_1 + 5 \cdot \gamma_1
= 5 \cdot \alpha_1 + 4 \cdot \beta_1 + 0 \cdot \gamma_1$. We have three equations and three unknowns. We find $\alpha_1 = \beta_1 = \gamma_1 = \frac{1}{3}$. Since the game is symmetric, this answer is not surprising. 

We do not need to check Step 3 because $C_{i} \backslash D_{i}$ is empty for $i = 1, 2$.
\end{enumerate}



To conclude we have one Nash equilibrium. It corresponds to
\begin{equation*}
    \squared{\dfrac{1}{3} \parent{x_1 + y_1 + z_1}, \dfrac{1}{3} \parent{x_2 + y_2 + z_2}}.    
\end{equation*}

The corresponding payoffs are $w = \parent{3,3}$. This concludes the exercise.
    \end{enumerate}
\end{solution}


\section{}
Compute all Nash equilibria of the following 3-player game:

	\begin{center}
		\begin{tabular}[h!]{l|ccc}
			& \Large{$x_3$} && \Large{$y_3$} \\
			&
			\begin{tabular}[h!]{cccccc}
				\hline
				& \Large{$x_2$} &&& \Large{$y_2$} & 
			\end{tabular}
			&&
			\begin{tabular}[h!]{cccccc}
				\hline
				& \Large{$x_2$} &&& \Large{$y_2$} & 
			\end{tabular} 
			\\[.2cm]
			\hline
			\\[-.4cm]
			\begin{tabular}[h!]{l}
				\Large{$x_1$} \\ \Large{$y_1$}
			\end{tabular}
			&
			\begin{tabular}[h!]{ccccc}
				& \Large{0,0,0} && \Large{6,5,4} & \\ 
				& \Large{5,4,6} && \Large{0,0,0} & 
			\end{tabular}
			&&
			\begin{tabular}[h!]{ccccc}
				& \Large{4,6,5} && \Large{0,0,0} & \\ 
				& \Large{0,0,0} && \Large{0,0,0} & 
			\end{tabular}
		\end{tabular} 
	\end{center}
	
In this game, the players receive money only if exactly one of them plays the action $y$.

\begin{solution}
Unfortunately we don't have any strongly dominated strategies in this game so we have a lot of supports to check.
However, we will heavily rely on the symmetry of the game.

\begin{enumerate} [label=\Alph*. ]
	\item \textbf{Pure strategy VS. Pure strategy VS. Pure Strategy.} \\
	      Using the concept of best response, we immediately find three equilibria, namely:
	      \begin{enumerate}
	      	\item The equilibrium $\parent{x_1, y_2, x_3}$ gives $w = \parent{6,5,4}$.
	      	\item The equilibrium $\parent{x_1, x_2, y_3}$ gives $w = \parent{4,6,5}$.
	      	\item The equilibrium $\parent{y_1, x_2, x_3}$ gives $w = \parent{5,4,6}$.
	      \end{enumerate}
	      	        
	      The fourth equilibrium which is not evident to find is $\parent{y_1, y_2, y_3}$, which gives $w = \parent{0,0,0}$.
	      	        
	\item \textbf{Pure strategy VS. Pure strategy VS. Randomized strategy.} \\
	      We take $\bracket{x_1} \times \bracket{x_2} \times \bracket{x_3, y_3}$. We find no equilibrium.
	      If we take $\bracket{y_1} \times \bracket{x_2} \times \bracket{x_3, y_3}$, then we find no equilibrium either.
	      The same conclusion follows for $\bracket{x_1} \times \bracket{y_2} \times \bracket{x_3, y_3}$
	      and $\bracket{y_1} \times \bracket{y_2} \times \bracket{x_3, y_3}$.
	      	          
	\item \textbf{Pure strategy VS. Randomized strategy VS. Pure strategy.} \\
	      By symmetry, there we are no equilibria with supports of this kind.
	      	      
	\item \textbf{Randomized strategy VS. Pure strategy VS. Pure strategy.} \\
	      Again, by symmetry, there we are no equilibria with supports of this kind.
	      	        
	\item \textbf{Pure strategy VS. Randomized strategy VS. Randomized strategy.} \\
	      We take $\bracket{x_1} \times \bracket{x_2, y_2} \times \bracket{x_3, y_3}$.
	      Hence, we define $\sigma_{1} \parent{x_1} = 1$.
	      We also set $\sigma_{2} \parent{x_2} = \beta$ and $\sigma_{2} \parent{y_2} = 1 - \beta$.
	      Finally, we define $\sigma_{3} \parent{x_3} = \gamma$ and $\sigma_{3} \parent{y_3} = 1 - \gamma$.
	      	              
	      	              
	      In order to find $\beta$, we look at Step 2 for player 3. We find
	      \begin{equation*}
	      	w_3 = \sum_{c_{-3} \in D_{-3}} \parent{\prod_{j \in \bracket{1,2}} \sigma_{j} \parent{c_{j}} } \cdot u_{3} \parent{c_{-3}, d_3} \ \forall \ d_{3} \in \bracket{x_3, y_3}
	      \end{equation*}
	      	              
	      where $c_{-3} = \parent{c_1, c_2}$ and $D_{-3} = D_{1} \times D_{2} = \bracket{x_1} \times \bracket{x_2, y_2}$.
	      Now we will compute the sum.
	      For $d_3 = x_3$, we find
	      \begin{align*}
	      	w_3
	      	  & = \sigma_{1} \parent{x_{1}} \cdot \sigma_{2} \parent{x_{2}} \cdot u_{3} \parent{x_1, x_2, x_3} 
	      	+  \sigma_{1} \parent{x_{1}} \cdot \sigma_{2} \parent{y_{2}} \cdot u_{3} \parent{x_1, y_2, x_3} \\
	      	  & = 1 \cdot \beta \cdot 0                                                                        
	      	+  1 \cdot \parent{1 - \beta} \cdot 4 \\
	      	  & = \parent{1 - \beta} \cdot 4.                                                                  
	      \end{align*}
	      	              
	      For $d_3 = y_3$, we find
	      \begin{align*}
	      	w_3
	      	  & = \sigma_{1} \parent{x_{1}} \cdot \sigma_{2} \parent{x_{2}} \cdot u_{3} \parent{x_1, x_2, y_3} 
	      	+  \sigma_{1} \parent{x_{1}} \cdot \sigma_{2} \parent{y_{2}} \cdot u_{3} \parent{x_1, y_2, y_3} \\
	      	  & = 1 \cdot \beta \cdot 5                                                                        
	      	+  1 \cdot \parent{1 - \beta} \cdot 0 \\
	      	  & = \beta \cdot 5.                                                                               
	      \end{align*}
	      The two expressions for $w_3$ must coincide, which is true if and only if $\parent{1 - \beta} \cdot 4 = \beta \cdot 5$, or again, if $\beta = \frac{4}{9}$.
	      Therefore we obtain $w_3 = \frac{20}{9} = \frac{220}{99}$.
	      	              
	      	              
	      	              
	      In order to find $\gamma$, we look at Step 2 for player 2. We find
	      \begin{equation*}
	      	w_2 = \sum_{c_{-2} \in D_{-2}} \parent{\prod_{j \in \bracket{1,3}} \sigma_{j} \parent{c_{j}} } \cdot u_{2} \parent{c_{-2}, d_2} \ \forall \ d_{2} \in \bracket{x_2, y_2}
	      \end{equation*}
	      	              
	      where $c_{-2} = \parent{c_1, c_3}$ and $D_{-2} = D_{1} \times D_{3} = \bracket{x_1} \times \bracket{x_3, y_3}$.
	      Now we will compute the sum.
	      For $d_2 = x_2$, we find
	      \begin{align*}
	      	w_2
	      	  & = \sigma_{1} \parent{x_{1}} \cdot \sigma_{3} \parent{x_{3}} \cdot u_{2} \parent{x_1, x_2, x_3} 
	      	+  \sigma_{1} \parent{x_{1}} \cdot \sigma_{3} \parent{y_{3}} \cdot u_{2} \parent{x_1, x_2, y_3} \\
	      	  & = 1 \cdot \gamma \cdot 0                                                                       
	      	+  1 \cdot \parent{1 - \gamma} \cdot 6 \\
	      	  & = \parent{1 - \gamma} \cdot 6.                                                                 
	      \end{align*}
	      	              
	      For $d_2 = y_2$, we find
	      \begin{align*}
	      	w_2
	      	  & = \sigma_{1} \parent{x_{1}} \cdot \sigma_{3} \parent{x_{3}} \cdot u_{2} \parent{x_1, y_2, x_3} 
	      	+  \sigma_{1} \parent{x_{1}} \cdot \sigma_{3} \parent{y_{3}} \cdot u_{2} \parent{x_1, y_2, y_3} \\
	      	  & = 1 \cdot \gamma \cdot 5                                                                       
	      	+  1 \cdot \parent{1 - \gamma} \cdot 0 \\
	      	  & = \gamma \cdot 5.                                                                              
	      \end{align*}
	      The two expressions for $w_2$ must coincide, which is true if and only if $\parent{1 - \gamma} \cdot 6 = \gamma \cdot 5$, or again, if $\gamma = \frac{6}{11}$.
	      Therefore we obtain $w_2 = \frac{30}{11} = \frac{270}{99}$.
	      	              
	      \vspace{5mm}
	      	              
	      	          
	      	              
	      Now we will check Step 3 for player 1. First we note that
	      \begin{equation*}
	      	w_1 = \sum_{c_{-1} \in D_{-1}} \parent{\prod_{j \in \bracket{2,3}} \sigma_{j} \parent{c_{j}} } \cdot u_{1} \parent{c_{-1}, d_1} \ \forall \ d_{1} \in \bracket{x_1}
	      \end{equation*}
	      	              
	      where $c_{-1} = \parent{c_2, c_3}$ and $D_{-1} = D_{2} \times D_{3} = \bracket{x_2, y_2} \times \bracket{x_3, y_3}$.
	      	              
	      Computing the four (!) terms of the sum, we find
	      \begin{align*}
	      	w_1
	      	  & = \sigma_{2} \parent{x_{2}} \cdot \sigma_{3} \parent{x_{3}} \cdot u_{1} \parent{x_1, x_2, x_3}  
	      	+  \sigma_{2} \parent{x_{2}} \cdot \sigma_{3} \parent{y_{3}} \cdot u_{1} \parent{x_1, x_2, y_3} \\
	      	  & +  \sigma_{2} \parent{y_{2}} \cdot \sigma_{3} \parent{x_{3}} \cdot u_{1} \parent{x_1, y_2, x_3} 
	      	+  \sigma_{2} \parent{y_{2}} \cdot \sigma_{3} \parent{y_{3}} \cdot u_{1} \parent{x_1, y_2, y_3} \\
	      	  & = \beta \cdot \gamma \cdot 0                                                                    
	      	+  \beta \cdot \parent{1 - \gamma} \cdot 4
	      	+  \parent{1 - \beta} \cdot \gamma \cdot 6
	      	+  \parent{1 - \beta} \cdot \parent{1 - \gamma} \cdot 0 \\
	      	  & = \dfrac{4}{9} \cdot \dfrac{5}{11} \cdot 4                                                      
	      	+ \dfrac{5}{9} \cdot \dfrac{6}{11} \cdot 6 \\
	      	  & = \dfrac{80 + 180}{99} = \dfrac{260}{99}.                                                       
	      \end{align*}
	      	              
	      	              
	      	              
	      For $e_1 = y_1 \in C_1 \backslash D_{1}$, we find
	      \begin{align*}
	      	w_1
	      	  & \geq \sum_{c_{-1} \in D_{-1}} \parent{\prod_{j \in \bracket{2,3}} \sigma_{j} \parent{c_{j}} } \cdot u_{1} \parent{c_{-1}, e_1} \\
	      	  & = \sigma_{2} \parent{x_{2}} \cdot \sigma_{3} \parent{x_{3}} \cdot u_{1} \parent{y_1, x_2, x_3}                                 
	      	+  \sigma_{2} \parent{x_{2}} \cdot \sigma_{3} \parent{y_{3}} \cdot u_{1} \parent{y_1, x_2, y_3} \\
	      	  & +  \sigma_{2} \parent{y_{2}} \cdot \sigma_{3} \parent{x_{3}} \cdot u_{1} \parent{y_1, y_2, x_3}                                
	      	+  \sigma_{2} \parent{y_{2}} \cdot \sigma_{3} \parent{y_{3}} \cdot u_{1} \parent{y_1, y_2, y_3} \\
	      	  & = \beta \cdot \gamma \cdot 5                                                                                                   
	      	+  \beta \cdot \parent{1 - \gamma} \cdot 0
	      	+  \parent{1 - \beta} \cdot \gamma \cdot 0
	      	+  \parent{1 - \beta} \cdot \parent{1 - \gamma} \cdot 0 \\
	      	  & = \dfrac{4}{9} \cdot \dfrac{6}{11} \cdot 5                                                                                     
	      	= \dfrac{120}{99}.
	      \end{align*}
	      	              
	      Indeed, we have $\dfrac{260}{99} \geq \dfrac{120}{99}$.
	      	              
	      \vspace{5mm}
	      	              
	      Putting everything together, we find that the equilibrium is
	      \begin{equation*}
	      	\parent{x_1,
	      		\dfrac{4}{9} x_2 + \dfrac{5}{9} y_2,
	      		\dfrac{6}{11} x_3 + \dfrac{5}{11} y_3}
	      \end{equation*}
	      	              
	      and the associated payoffs are $w = \dfrac{1}{99} \cdot \parent{260, 270, 220}$.
	      	              
	      	              
	      	              
	\item \textbf{Randomized strategy VS. Pure strategy VS. Randomized strategy.} \\
	      By symmetry, we know that we can relabel the number of the players. We set $1 \rightarrow 2$, $2 \rightarrow 3$ and $3 \rightarrow 1$. We find that the equilibrium is
	      \begin{equation*}
	      	\parent{\dfrac{6}{11} x_1 + \dfrac{5}{11} y_1,
	      		x_2,
	      		\dfrac{4}{9} x_3 + \dfrac{5}{9} y_3}
	      \end{equation*}
	      	              
	      and the associated payoffs are $w = \dfrac{1}{99} \cdot \parent{220, 260, 270}$.
	      	                  
	      	          
	\item \textbf{Randomized strategy VS. Randomized strategy VS. Pure strategy.} \\
	      By symmetry, we know that we can again relabel the number of the players. We start from case E. We set $1 \rightarrow 3$, $2 \rightarrow 1$ and $3 \rightarrow 2$. We find that the equilibrium is
	      \begin{equation*}
	      	\parent{\dfrac{4}{9} x_1 + \dfrac{5}{9} y_1
	      		\dfrac{6}{11} x_2 + \dfrac{5}{11} y_2,
	      	x_3}
	      \end{equation*}
	      	              
	      and the associated payoffs are $w = \dfrac{1}{99} \cdot \parent{270, 220, 260}$.
	      	                  
	\item \textbf{Randomized strategy VS. Randomized strategy VS. Randomized strategy.} \\
	      The only support we can take is $\bracket{x_1, y_1} \times \bracket{x_2, y_2} \times \bracket{x_3, y_3}$.
	      Hence, we define $\sigma_{1} \parent{x_1} = \alpha$ and $\sigma_{1} \parent{y_1} = 1 - \alpha$.
	      We also set $\sigma_{2} \parent{x_2} = \beta$ and $\sigma_{2} \parent{y_2} = 1 - \beta$.
	      Finally, we define $\sigma_{3} \parent{x_3} = \gamma$ and $\sigma_{3} \parent{y_3} = 1 - \gamma$.  
	      By symmetry, we guess that $\alpha = \beta = \gamma$.
	      	      
	      	      
	      	      
	      	      
	      First, we look at Step 2 for player 1. We find
	      \begin{equation*}
	      	w_1 = \sum_{c_{-1} \in D_{-1}} \parent{\prod_{j \in \bracket{2,3}} \sigma_{j} \parent{c_{j}} } \cdot u_{1} \parent{c_{-1}, d_1} \ \forall \ d_{1} \in \bracket{x_1, y_1}
	      \end{equation*}
	      	              
	      where $c_{-1} = \parent{c_2, c_3}$ and $D_{-1} = D_{2} \times D_{3} = \bracket{x_2, y_2} \times \bracket{x_3, y_3}$.
	      Now we will compute the sum.
	      For $d_1 = x_1$, we find
	      \begin{align*}
	      	w_1
	      	  & = \sigma_{2} \parent{x_{2}} \cdot \sigma_{3} \parent{x_{3}} \cdot u_{1} \parent{x_1, x_2, x_3}   
	      	+   \sigma_{2} \parent{x_{2}} \cdot \sigma_{3} \parent{y_{3}} \cdot u_{1} \parent{x_1, x_2, y_3} \\
	      	  & +   \sigma_{2} \parent{y_{2}} \cdot \sigma_{3} \parent{x_{3}} \cdot u_{1} \parent{x_1, y_2, x_3} 
	      	+   \sigma_{2} \parent{y_{2}} \cdot \sigma_{3} \parent{y_{3}} \cdot u_{1} \parent{x_1, y_2, y_3} \\
	      	  & = \beta \cdot \gamma \cdot 0                                                                     
	      	+   \beta \cdot \parent{1 - \gamma} \cdot 4
	      	+   \parent{1 - \beta} \cdot \gamma \cdot 6
	      	+   \parent{1 - \beta} \cdot \parent{1 - \gamma} \cdot 0 \\
	      	  & = \alpha \cdot \parent{1 - \alpha} \cdot 4                                                       
	      	+   \parent{1 - \alpha} \cdot \alpha \cdot 6
	      	= 10 \alpha \cdot \parent{1 - \alpha}.
	      \end{align*}
	      	              
	      For $d_1 = y_1$, we find
	      \begin{align*}
	      	w_1
	      	  & = \sigma_{2} \parent{x_{2}} \cdot \sigma_{3} \parent{x_{3}} \cdot u_{1} \parent{y_1, x_2, x_3}   
	      	+   \sigma_{2} \parent{x_{2}} \cdot \sigma_{3} \parent{y_{3}} \cdot u_{1} \parent{y_1, x_2, y_3} \\
	      	  & +   \sigma_{2} \parent{y_{2}} \cdot \sigma_{3} \parent{x_{3}} \cdot u_{1} \parent{y_1, y_2, x_3} 
	      	+   \sigma_{2} \parent{y_{2}} \cdot \sigma_{3} \parent{y_{3}} \cdot u_{1} \parent{y_1, y_2, y_3} \\
	      	  & = \beta \cdot \gamma \cdot 5                                                                     
	      	+   \beta \cdot \parent{1 - \gamma} \cdot 0
	      	+   \parent{1 - \beta} \cdot \gamma \cdot 0
	      	+   \parent{1 - \beta} \cdot \parent{1 - \gamma} \cdot 0 \\
	      	  & = \alpha \cdot \alpha \cdot 5.                                                                   
	      \end{align*}
	      	      
	      	      
	      The two expressions for $w_1$ must coincide, which is true if and only if $10 \alpha \cdot \parent{1 - \alpha} = 5 \cdot \alpha^{2}$, or again, if $\alpha = \frac{2}{3}$.
	      Therefore we obtain $w_1 = \frac{20}{9}$.
	      	   
	      Hence, by the symmetry of the problem, we find that the equilibrium is
	      \begin{equation*}
	      	\parent{\dfrac{2}{3} x_1 + \dfrac{1}{3} y_1,
	      		\dfrac{2}{3} x_2 + \dfrac{1}{3} y_2,
	      		\dfrac{2}{3} x_3 + \dfrac{1}{3} y_3}
	      \end{equation*}
	      	              
	      and the associated payoffs are $w = \dfrac{1}{9} \cdot \parent{20, 20, 20}$.     
	      	            	              
      	              
\end{enumerate}


Hence, putting everything together, we have found eight (!) equilibria for this three-player game.
\end{solution}

\section{}
A Nash equilibrium is called \emph{strong} if there exists no subset of players that could all benefit from deviating together from the equilibrium, assuming that players that are not part of this subset play their equilibrium strategy. More formally, a randomized strategy $\sigma$ is a strong Nash equilibrium iff for any randomized strategy $\tau$ different from $\sigma$, there exists a player $i$ such that $\tau_i \neq \sigma_i$ and $u_i(\tau) \leq u_i(\sigma)$.
\begin{enumerate}
	\item[a.] Find a game with at least three players for which such a strong Nash equilibrium exists.
	\item[b.] Show that the game defined in exercise 2. does not allow any strong Nash equilibrium.
\end{enumerate}

\section{}
A new oil deposit has been discovered in the North Sea. Two large buyers will play the auction. 
The area of the deposit is divided into three parts of respective area $A_0, A_1, A_2$. Each part contains a certain fraction of oil $\tilde{x}_0, \tilde{x}_1$ and $\tilde{x}_2$ under the ground. It is assumed that these fractions are i.i.d. and follow a homogeneous distribution so the $\tilde{x}_i$'s are uniformly distributed in~$[0, 1]$. In the end, the total value of the deposit is given by $A_0 \tilde{x}_0 + A_1 \tilde{x}_1 + A_2 \tilde{x}_2$. The areas of the three parts as well as the amount of oil contained in the part with index~0 are common knowledge to the buyers. Moreover, Buyer~1 (resp. Buyer~2) was able to measure the fraction of oil present in the part with index~1 (resp. index~2).

Based on their knowledge, the two buyers should simultaneously make an offer for the deposit, $c_1$ and $c_2$ respectively. The buyer who bets the highest amount wins the deposit at the price he offered (in case of a tie, a coin is flipped to choose the winner).
\begin{enumerate}
	\item[a.] Model this game as a Bayesian game.
\end{enumerate}
It is known that there exists a unique Bayesian equilibrium to this game of the form $\{ \ c_1 = \alpha_1 \tilde{x}_0 + \beta_1 \tilde{x}_1, \ c_2 = \alpha_2 \tilde{x}_0 + \beta_2 \tilde{x}_2 \ \}$.
\begin{enumerate}
	\item[b.] Find this equilibrium.
	\item[c.] Let us assume that $A_0 = A_1 = A_2 = 100$ and that Buyer~1 observed $\tilde{x}_0 = 0$ and $\tilde{x}_1 = 0.01$. What should be his bet? Does this result sound reasonable? Would it not be safer to increase the offer a bit? Why?
\end{enumerate}

%%The intrinsic value of the deposit is common knowledge to both players (it is the number of barrels that is possible to extract). However, 
%None of the buyers know the exact amount of oil contained in the deposit and they can only estimate it based on public information and private measures. In the end, it is known that the value of the deposit in the eyes of both buyers is given by $A_0 \tilde{x}_0 + A_1 \tilde{x}_1 + A_2 \tilde{x}_2$, where $A_0, A_1$ and $A_2$ are positive constants known by all and $\tilde{x}_0, \tilde{x}_1$ and $\tilde{x}_2$ are random variables with uniform distribution on the interval $[0, 1]$. In addition, at the time of the auction, Buyer~1 will have seen $\tilde{x}_0$ and $\tilde{x}_1$ while Buyer~2 will have seen $\tilde{x}_0$ and $\tilde{x}_2$. Based on their knowledge, they then simultaneously offer $c_1$ and $c_2$ respectively. The buyer who bets the highest amount wins the deposit at the price he offered (in case of a tie, a coin is flipped to choose the winner). Therefore, the utility function of the players is given by:
%\begin{align*}
%	u_i(c_1, c_2, (\tilde{x}_0, \tilde{x}_1), (\tilde{x}_0, \tilde{x}_2)) = 
%	\begin{cases}
%		\hspace{.2cm} A_0 \tilde{x}_0 + A_1 \tilde{x}_1 + A_2 \tilde{x}_2 - c_i & \text{if } c_i > c_j, \\
%		(A_0 \tilde{x}_0 + A_1 \tilde{x}_1 + A_2 \tilde{x}_2 - c_i) / 2 & \text{if } c_i = c_j, \\
%		0 & \text{if } c_i < c_j.
%	\end{cases}
%\end{align*}
%%\begin{equation*}
%%	u_i(c_1, c_2, (\tilde{x}_0, \tilde{x}_1), (\tilde{x}_0, \tilde{x}_2)) = 
%%	\begin{cases}
%%		\hspace{.2cm} A_0 \tilde{x}_0 + A_1 \tilde{x}_1 + A_2 \tilde{x}_2 - c_i 		\hspace{1.2cm} \text{if } c_i > c_j \\
%%		(A_0 \tilde{x}_0 + A_1 \tilde{x}_1 + A_2 \tilde{x}_2 - c_i) / 2 						\hspace{ .5cm} \text{if } c_i = c_j \\
%%		0 																																					\hspace{6.45cm} \text{if } c_i < c_j.
%%	\end{cases}
%%\end{equation*}
%It is known that there exists a unique Bayesian equilibrium to this game of the form $\{ \ c_1 = \alpha_1 \tilde{x}_0 + \beta_1 \tilde{x}_1, \ c_2 = \alpha_2 \tilde{x}_0 + \beta_2 \tilde{x}_2 \ \}$.
%\begin{enumerate}
%	\item[a.] Find this equilibrium.
%	\item[b.] Let us assume that $A_0 = A_1 = A_2 = 100$ and that Buyer~1 observed $\tilde{x}_0 = 0$ and $\tilde{x}_1 = 0.01$. What should be his bet? Does this result sound reasonable? Would it not be safer to increase the offer a bit? Why?
%\end{enumerate}

%\newpage

\thex{}
Let $\Gamma^1$ be the game obtained from $\Gamma$ by removing all its strongly dominated strategies. Show that $\sigma$ is a Nash equilibrium of $\Gamma^1$ iff $\sigma$ is a Nash equilibrium of $\Gamma$.

\thex{}
Let $\Gamma^2$ be the game obtained from $\Gamma$ by removing all its weakly dominated strategies. Show that if $\sigma$ is a Nash equilibrium of $\Gamma^2$, then $\sigma$ is also a Nash equilibrium of $\Gamma$. Show that the converse statement is not true.

\thex{}
Let $\Gamma$ be a 2-player zero-sum game in strategic form. Show that the set $$\{ \sigma_1 \, | \, \sigma \text{ is an equilibrium of } \Gamma \}$$ is a convex subset of $\Delta(C_1)$.

\thex{}
Consider a game $\Gamma(N,C,u)$. Given a player $i \in N$, and a randomized strategy profile of $i$ opponents, $\sigma_{-i} \in \Delta C_{-i}$, Nash' Lemma states that there is always a best response from $i$ to $\sigma_{-i}$ which is a pure strategy in $C_i$. 
\begin{itemize}
\item Explain all of the notations above.
\item Formalize Nash lemma mathematically.
\item Provide a proof of the Lemma. Hint: Remember that the strategy $\sigma_{-i}$ is fixed when considering the concept of best responses. 
\end{itemize} 

\thex{} Let $Y$ and $Z$ be compact (closed and bounded), convex and non-empty sets of finite dimension. 
%
%Let $g : Y \times Z \rightarrow \mathbb{R}$ be a continuous function that is linear in $y$.
%
%Let $F : Z \rightarrow \rightarrow Y$ be such that $F(z) = \arg \max\limits_{y \in Y} g(y, z)$. \\ (Note that $F$ is a point-set matching, hence the $\rightarrow \rightarrow$ notation.)
%
%Soit $F : Z \rightarrow \rightarrow Y$ tel que $F(z) = \arg \max\limits_{y \in Y} g(y, z)$. \\ (Notez que $F$ est une correspondance point-ensemble, d'où la notation $\rightarrow \rightarrow$.)
%\begin{enumerate}
%	\item[a.] Montrez que $F$ est hémicontinue supérieurement.
%	\item[b.] Montrez que $F$ est convexe.
%\end{enumerate}
%Ceci conclut la démonstration du théorème de Nash telle qu'exposée dans le livre ``Game Theory : Analysis of Conflict'' de R. Myerson.

\end{document}
