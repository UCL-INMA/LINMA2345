\begin{itemize}
  \item[a.] The decision that we will do during the first turn of the prisoner dilemma depends of the one done during the second turn,
  that depends of the one of the third turn,etc. As we have a finite number of games, we will check the rationnal decison done at the last turn,
  the game $N$.
  In this case, the past has no impact and the players will both defect $(D,d)$. Now that we know that, we can done a decision $(D,d)$
  for the game $N-1$. Recursively, we can go back to turn 1 and players will both defect $(D,d)$.
  As we know when the games are over, everything can be planned in advance.
  \item[b.] We add randomness: a probability $\alpha$ that we will continue to play, and a probability $(1-\alpha)$ that we stop playing.
  The set of player is $N = \{1,2\}$, the set of state of the wrolds is $\Theta = \{\text{Play},\text{Stop}\}$, the initial condition probability
  $q$ is $p(\text{Play})= 1$, and the transition between states probabilities $p$ is $p(\text{Play}|\text{Play},d)=\alpha$,
  $p(\text{Stop}|\text{Play},d)=1-\alpha$ and $p(\text{Stop}|\text{Stop},d)=1$. 
  (we could possibly add states and signals.)
  \item[c.] If players plays Confess or Deny randomly with $p=\frac{1}{2}$ each, there are 4 possibilities with each a
  probability $p=\frac{1}{4}$ that it's happen.
  By symmetry, the average payoff of each player is then : $\frac{1}{4}(5+1+0+6) = 3$.
  By taking in account the discount factor for the repeated games, the payoff will be : $3 + 3\delta + 3\delta^2 + 3\delta^3 +... =
  \frac{3}{1-\delta}$ (Geometric series).
  (Easier notation: $(1-\delta)(3+3\delta+3\delta^2+...) = (1-\delta)\frac{3}{1-\delta} = 3$) 
\end{itemize}
