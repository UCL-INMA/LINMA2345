\begin{enumerate}[label=\alph*.]

    %%%%%%%%%%%%%%%%%%
    %%%%%%% a %%%%%%%%
    %%%%%%%%%%%%%%%%%%

    \item A bayesian game is a tuple $\Gamma^b =(N, C, T, p, u)$. In this situation:
        \begin{enumerate}
            \item $N = \{1,2\}$,
            \item $C = C_1\times C_2$ where $C_i=\{0, \dotsc, 100\}$ for $i=1,2$,
            \item $T = T_1\times T_2$ where $T_i=\{1, \dotsc, 100\}$ for $i=1,2$,
            \item $p = (p_1:T_1 \to \Delta(T_2), p_2:T_2\to \Delta(T_1))$. $\forall i\in T_1,j\in T_2$ $p_1(t_2=j\vert t_1=i) = p_2(t_1=i\vert t_2=j)= \frac{1}{100}$,
            \item $u = (u_1, u_2)$ where $u_i :C\times T\to \R$ for $i=1, 2$. To write the expected payoff function, we make the following observations: 
            \begin{enumerate}
                \item The winner pays the opponents bid.
                \item In case of a draw, a coin is flipped.
                \item If you lose, the payoff is zero.
            \end{enumerate}
            \[
            u_1\bigl((c_1,c_2),(t_1,t_2)\bigr) =
            \begin{cases}
            t_1 - c_2 & \text{if } c_1 > c_2, \\
            \frac{1}{2}(t_1 - c_2) & \text{if } c_1 = c_2, \\
            0 & \text{if } c_1 < c_2.
            \end{cases}
            \]
            \[
            u_2\bigl((c_1,c_2),(t_1,t_2)\bigr) =
            \begin{cases}
            t_2 - c_1 & \text{if } c_2 > c_1, \\
            \frac{1}{2}(t_2 - c_1) & \text{if } c_2 = c_1, \\
            0 & \text{if } c_2 < c_1.
            \end{cases}
            \]
        \end{enumerate}

    %%%%%%%%%%%%%%%%%%
    %%%%%%% b %%%%%%%%
    %%%%%%%%%%%%%%%%%%

    \item For the type-agent formulation, we consider 1 player per type. As there are 100 types per player in the bayesian formulation, there will be 200 players in the type-agent formulation. Each of those players will have 101 actions, the same as in the bayesian formulation.
    
    %%%%%%%%%%%%%%%%%%
    %%%%%%% c %%%%%%%%
    %%%%%%%%%%%%%%%%%%

    \item In a second-price sealed-bid auction, bidding truthfully $b=t_1$ weakly dominates any other bid $b\neq t_1$. \\
        To see this, fix a type $t_1$ for Player~1 and consider any bid $b\neq t_1$.  
        We compare the payoff obtained by bidding $b$ and by bidding $t_1$, for every possible bid $b_2$ of Player~2.

        \medskip

        \noindent\textbf{Case 1: $b<t_1$.}

        \begin{itemize}
            \item If $b_2<b<t_1$, Player~1 wins under both bids and pays $b_2$, hence the payoff is $t_1-b_2$ in both cases.
            
            \item If $b_2=b<t_1$, bidding $b$ leads to a tie and an expected payoff $\frac{1}{2}(t_1-b_2)$, whereas bidding $t_1$ leads to a sure win with payoff $t_1-b_2$. Thus bidding $t_1$ yields a strictly higher payoff.
            
            \item If $b<b_2<t_1$, bidding $b$ leads to a loss and payoff $0$, while bidding $t_1$ leads to a win with payoff $t_1-b_2>0$. Hence bidding $t_1$ is strictly better.
            
            \item If $b<t_1\le b_2$, Player~1 loses under both bids and obtains payoff $0$ in both cases.
        \end{itemize}

        In all subcases, bidding $b=t_1$ is at least as good as any $b<t_1$, and strictly better in some cases.
        \medskip

        \noindent\textbf{Case 2: $b>t_1$.}

        \begin{itemize}
            \item If $b_2\leq t_1<b$, Player~1 wins under both bids and pays $b_2$, hence the payoff is $t_1-b_2$ in both cases.
        
            \item If $t_1<b_2<b$, bidding $b$ leads Player~1 to win and pay $b_2$, yielding payoff $t_1-b_2<0$, whereas bidding $t_1$ leads to a loss and payoff $0$. Thus bidding $t_1$ is strictly better.
            
            \item If $t_1<b<b_2$, Player~1 loses under both bids and obtains payoff $0$ in both cases.
        \end{itemize}

        Again, bidding $b=t_1$ is at least as good as any $b>t_1$, and strictly better in some cases. 

        In a first-price sealed-bid auction, bidding truthfully always lead to a zero payoff. Therefore, it is not the optimal strategy anymore. Finding the optimal bid is not as easy, but it will obviously be less than the type of the player.

\end{enumerate}