\begin{enumerate}[label=\alph*.]
  
  
  %%%%%%%%%%%%%%%%%%%%%%%%
  %%%%%%% a %%%%%%%%%%%%%%
  %%%%%%%%%%%%%%%%%%%%%%%%
  \item We will apply Bayes' formula, i.e.
  \begin{equation*}
    \mathbb{P} \squared{A | B} = \dfrac{\mathbb{P} \squared{A \cap B}}{\mathbb{P} \squared{B}}.
  \end{equation*}
  
  We need to prove that, for all $k >0$, we have
  \begin{equation*}
    \mathbb{P} \squared{Y_i = X+1 | Y_i = k} > \dfrac{1}{2}.
  \end{equation*}
  
  Applying Bayes's rule, we have 
  \begin{align*}
      \mathbb{P} \squared{Y_i = X+1 | Y_i = k}
      &= \dfrac{\mathbb{P} \squared{Y_i = X+1 = k}}{\mathbb{P} \squared{Y_i = k}}.
  \end{align*}
  
  Let us compute numerator and denominator independently, for clarity. We find
  \begin{align*}
      \mathbb{P} \squared{Y_i = X+1 = k}
      &= \mathbb{P} \squared{X+1 = k}
      = \mathbb{P} \squared{X = k-1}
      = 0.9 ^{k-1} \times 0.1,
  \end{align*}
  
  and
  \begin{align*}
      \mathbb{P} \squared{Y_i = k}
      &= \mathbb{P} \squared{X = k} + \mathbb{P} \squared{X+1 = k} \\
      &= \mathbb{P} \squared{X = k} + \mathbb{P} \squared{X = k-1} \\
      &= 0.9 ^{k} \times 0.1 + 0.9 ^{k-1} \times 0.1.
  \end{align*}
  
  Putting everything together,  and using the fact that $k >0$, we obtain
  \begin{align*}
      \mathbb{P} \squared{Y_i = X+1 | Y_i = k}
      &= \dfrac{0.9 ^{k-1} \times 0.1}{0.9 ^{k} \times 0.1 + 0.9 ^{k-1} \times 0.1} \\
      &= \dfrac{0.9 ^{-1}}{1 + 0.9 ^{-1}} \\
      &= \dfrac{1}{0.9 + 1} \\
      &= \dfrac{10}{19} > \dfrac{1}{2}.
      \end{align*}
  
  The proof is complete.
  
  
  %%%%%%%%%%%%%%%%%%%%%%%%
  %%%%%%% b %%%%%%%%%%%%%%
  %%%%%%%%%%%%%%%%%%%%%%%%
  \item The words ``\textit{both players know that}'' can be repeated at most $k-1$ times.
    Indeed, the players receiving $\X$ only knows that $\X \ge k-1$ as he thinks he might have received $\X + 1$.
    The sample player then thinks that the other player might have received $k - 1$.
    In which case, he thinks that this other players thinks that he might have received $k - 2$.
    Continuing this reasoning $k$ times, the player thinks the other player thinks that he thinks that the other player thinks ... that he (or the other player if $k$ is odd) might have received 0.
    As a player receiving 0 has lost, the reasoning cannot be repeated $k$ times.
    
    Even if it is not required in the statement, we also prove below that the sentence can be repeated $k - 1$ times.
  
  We define two operators
  \begin{itemize}
    \item $\text{K} \parent{\cdot}$: both players \textit{know} that $\cdot$ 
    \item T: both players \textit{think} that they will win 
  \end{itemize}
  
  From exercise (a), we know that if $X \geq 1$, then T is true. We will need the following lemma.
  
  \paragraph{Lemma.} If $X \geq k + 1$, then K $\parent{X \geq k}$ is true.
  
  We need to prove the following theorem.
  
  \paragraph{Theorem.} If $X = k > 1$, then $\text{K}^{k-1} \parent{\text{T}}$.
  
  Of course, we will prove that by assumption.
  
  \paragraph{Base case, $k=2$.} Trivial.
  \paragraph{Induction case.} Assume that the statement is true for $k$,
  i.e., assume that if $X = k > 1$, then $\text{K}^{k-1} \parent{\text{T}}$ is true.
  We have to show that the statement is also true for $k+1$, i.e., we have to show that if $X = k+1 > 1$,
  then $\text{K}^{k} \parent{\text{T}}$ is true.
  Using the above lemma, $X = k+1$ implies that $\text(K)(X \ge k)$.
  By the induction hypothesis, this implies $\text(K)(\text{K}^{k-1} \parent{\text{T}}) = \text{K}^{k} \parent{\text{T}}$.
  
  The proof is complete.
  
  
  %%%%%%%%%%%%%%%%%%%%%%%%
  %%%%%%% c %%%%%%%%%%%%%%
  %%%%%%%%%%%%%%%%%%%%%%%%
  \item We know that the event $X \geq 0$ is common knowledge. Can we say the same thing about the event $X \geq 1$, or not ?
  We define $\text{M} \parent{\cdot}$ to denote the words "there exists a player who think that, maybe, $\cdot$ is true". We have, for every event $A$ , and for every integer $t$, 
  \begin{equation*}
      \begin{cases}
        \neg \text{K} \parent{A} = \text{M} \parent{\neg A}, \\
        \neg \text{K}^{t} \parent{A} = \text{M}^{t} \parent{\neg A},
      \end{cases}
  \end{equation*}
  
  where $\neg A$ denotes the negation of event $A$. Suppose that if $X = k$, then $\text{M} \parent{X < k}$.
  We know that if $X < k$, then $\text{M} \parent{X < k-1}$. Recursively, we find
  \begin{align*}
      \text{M}^{k-1} \parent{X < 1}
      &= \text{M}^{k-1} \parent{ \neg X \geq 1}
      = \neg \text{K}^{k-1} \parent{X \geq 1}.
  \end{align*}
  
  This shows that the event $X \geq 1$ is not common knowledge.
  
  
\end{enumerate}
